\documentclass[../main.tex]{subfiles}
\begin{document}
\dfn{Mengen}{Nach Cantor ist eine Menge $M$ eine Zusammenfassung bestimmter, wohlunterschiedener Objekte unserer Anschauung oder unseres Denkens (welche die Elemente von M genannt werden) zu einem Ganzen.\\
        Diese Objekte heißen Elemente:
        \begin{align*}
            Bsp.: \:\:A:&=\{M,A,T,H,E,M,A,T,I,K\}\\
            \,&=\{M,A,H,T,E,A,I,K\}\\
            \,&=\{T,H,E,M,T,A,I,K\}
        \end{align*}
        Man schreibt $x\in A $, wenn $A$ eine Menge ist und $x$ ein Element von $A$ ist.\\
        Ist $x$ kein Element von A, so schriebt man $x\notin A$\\\\
        Ist $H(.)$ eine Aussagem die von einer Variable $x$ abhängt, dann gibt es eine Menge 
        $$A:=\{x|H(x)\}$$
        D.h. $x\in A\iff H(x)$ ist wahr\\

        $H(x):=\{x^2-3x+2=0\}$\\
        $B:=\{x|H(x)\}=\{1,2\}$\\}

\dfn{Gleichheit, leere Menge, Teilmenge}{\begin{enumerate}
                \item Zwei Mengen $A$ und $B$ sind gleich, wenn sie die selben Elemente enthalten.
                \item Die leere Menge ($\emptyset$) ist die eindeutige Menge, welche kein Elemnt enthält
                \item Teilmengen: Wenn alle Elemente von $A$ auch Elemente von B sind, dann ist A Teilmenge von B.\\
                $A\subseteq B$ bzw. $A\supseteq B$ für alle $x\in A$ folgt $x\in B$\\
                Bemerkung: $A=B\iff (A\subseteq B)\land (B\subseteq A)$
                \item Ist $A\subseteq B $ und $ A\neq B$, dann nennt man $A$ echte Teilmenge von $B$ $$A\subsetneq B$$
                \item Zwei Mengen sind disjunkt, falls $x\in A\land x\notin B$ oder $A \cap B = \emptyset$  
            \end{enumerate}}

\subsection{Operationen mit Mengen}
Seien $A,B$ Mengen
            \begin{itemize}
                \item Durchschnitt: $A\cap B :=\{x|(x\in A)\land (x\in B)\}$
                \item Vereinigung: $A\cup B :=\{x|(x\in A)\lor (x\in B)\}$
                \item Differenz/Komplement: $A\setminus B :=\{(x\in A)\land (x \notin B)\}$
                \subitem Ist $A\subseteq M: A^C=A^C_M=M\setminus A $
            \end{itemize}

\subsection{Regeln für Mengen}
Seien $A$, $B$, $C$, $M$ Mengen und $A, B \subseteq M$. Dann gilt:
            \begin{enumerate}
                \item $A \cap B = B \cap A$ und $A \cup B = B \cup A$.
                \item $(A \cap B) \cap C = A \cap (B \cap C)$ und $(A \cup B) \cup C = A \cup (B \cup C)$.
                \item $A \cap (B \cup C) = (A \cap B) \cup (A \cap C)$ und $A \cup (B \cap C) = (A \cup B) \cap (A \cup C)$.
                \item $(A \cap B)^C = (A \cap B)^C_M = A^C \cup B^C = A^C_M \cup B^C_M $\\
                $(A \cup B)^C = A^C \cap B^C$\\
                $M\setminus (A \cap B) = (M\setminus A) \cup (M\setminus B)$\\
                $M\setminus (A \cup B) = (M\setminus A) \cap (M\setminus B)$\\                         
            \end{enumerate}

\clm{$\cap$ ist kommutativ}{}{$A\cap B\iff B\cap A$}
\sol{
 \begin{align*}
     A\cap B &= \{x|x\in A \land x\in B\}\\
     \, &= \{x|x\in B \land c\in A\}\\
     \, &=B\cap A
 \end{align*}
 \qed
}\\
\clm {Distributivität von $\cap$ und $\cup$}{}{$A \cup (B\cap C)\iff (A\cup B) \cap (A\cup C)$}
\sol{
\begin{align*}
    x\in A \cup (B\cap C) &\iff x\in A \land x\in B\cap C\\
    \, &\iff x\in A\lor (x\in B\land x\in C) \\
    \, &\iff (x\in A \lor x\in B)\land (c\in A \lor x\in C)\\
    \, &\iff x\in A\cup B \land x\in A\cup B\\
    \, &\iff (A\cup B) \cap (A\cup C)\\
\end{align*}
    \qed
}\\
Die restlichen Beweise sind ähnlich


\dfn{Mengenfamilien}{
Sei $J$ beliebige Menge $J\neq \emptyset$\\
Eine Familie von Mengen (Mengenfamilie) ist gegeben durch $A_j$ fpr jeden $j\in J$\\
Schreibe:$$\{A_j\}_{j\in J}$$
}

\dfn{Schnitt und Vereinigungsmengen}{
Es kommt öfters vor, dass man eine Menge $I$ gegeben hat (Indexmenge genannt) und jedem Element $i \in I$ der Indexmenge wird eine Menge $A_i$ zugeordnet. So eine Zuordnung nennt man dann auch Mengenfamilie indiziert über $I$. In so einem Fall schreibt man dann auch:
$$\begin{aligned}
& \bigcap_{i \in I} A_i:=\left\{x \in M \mid \forall i \in I: x \in A_i\right\} \\
& \bigcup_{i \in I} A_i:=\left\{x \in M \mid \exists i \in I: x \in A_i\right\}
\end{aligned}$$
}

\section{Quantoren}
\textit{Siehe Lineare Algebra Skript}
\section{Kartesishes Produkt, Relationen und Äquivalenzrelationen}
\dfn{Kartesisches Produkt}{Sind $M$ und $N$ Mengen, und ist $m \in M$ und $n \in N$, so bezeichnet $(m, n)$ das geordnete Paar bestehend aus $m \in M$ und $n \in N$. Zwei solche Paare $\left(m_1, n_1\right)$ und $\left(m_2, n_2\right)$ sind nach Definition genau dann gleich, wenn $m_1=m_2$ und $n_1=n_2$. Man schreibt
$$
M \times N:=\{(x, y) \mid x \in M \wedge y \in N\}
$$
und nennt $M \times N$ das kartesische Produkt von $M$ und $N$.}

\end{document}